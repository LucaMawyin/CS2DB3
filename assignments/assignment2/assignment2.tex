%% Determines the type of document (including standard settings for layout).
\documentclass[letterpaper]{article}



%% Package to control the size of a page: the standard margins used by LaTeX are
%% very wide!
\usepackage[margin=1in]{geometry}


%% Packages add functionality to the document. The AMS packages are standard
%% packages to support various mathematical notations. AMS stands for American
%% Mathematical Society, the organization that maintains these packages.
\usepackage{amsmath,amsthm,amssymb}

%% Theorems-like environments using functionality provided by amsthm.
\theoremstyle{plain}
\newtheorem{theorem}{Theorem}[section]
    %% [section] at the end specifies that theorems should be numbered
    %% per-section: Section x starts with theorem-like x.1 and so on, ...
\newtheorem{proposition}[theorem]{Proposition}
    %% [theorem] in the middle specifies: use the same counter as the theorem
    %% environment: here we number all theorem-like environments consecutively.
\newtheorem{corollary}[theorem]{Corollary}
\newtheorem{lemma}[theorem]{Lemma}
\theoremstyle{definition}
\newtheorem{definition}[theorem]{Definition}
\theoremstyle{remark}
\newtheorem{example}[theorem]{Example}
\newtheorem{remark}[theorem]{Remark}


%% Support for nicely formatted tables.
\usepackage{booktabs}


%% Support for colors & colors in tables.
\usepackage[table]{xcolor}

% Seven colors safe for use color blindness.
% Colors taken from doi:10.1038/nmeth.1618.
\definecolor{cbOrange}{RGB}{230,159,0}
\definecolor{cbSkyBlue}{RGB}{86,180,233}
\definecolor{cbBluishGreen}{RGB}{0,158,115}
\definecolor{cbYellow}{RGB}{240,228,66}
\definecolor{cbBlue}{RGB}{0,114,178}
\definecolor{cbVermillion}{RGB}{213,94,0}
\definecolor{cbReddischPurple}{RGB}{204,121,167}


%% Notation used in this document.
\newcommand{\n}{\mathbf{n}} %% Num. Replicas.
\newcommand{\f}{\mathbf{f}} %% Num. Faulty Replicas.

%% Misc. Math notation.
\newcommand{\BigO}{\mathcal{O}}
\newcommand{\abs}[1]{\lvert #1 \rvert}
\newcommand{\AName}[1]{\textsc{#1}}
\newcommand{\Var}[1]{\texttt{#1}}


%% Algorithms.
\usepackage{algorithm}
\usepackage[noend]{algorithmic}
\newcommand{\GETS}{:=}


%% Formatting SI-units.
\usepackage{siunitx}
\sisetup{per-mode=symbol}


%% TikZ: for creating figures.
\usepackage{tikz}

%% Configuration for figures: Nicer arrows.
\usetikzlibrary{arrows.meta}
\tikzset{>=Stealth}


%% pgfplots: drawing plots using TikZ.
\usepackage{pgfplots}
%% Configuration for plots: Use color-blind friendly colors.
\pgfplotscreateplotcyclelist{cbSafeList}{
    very thick,solid,cbOrange,every mark/.append style={solid},mark=*\\
    very thick,solid,cbSkyBlue,every mark/.append style={solid},mark=*\\
    very thick,solid,cbBluishGreen,every mark/.append style={solid},mark=*\\
    very thick,solid,cbYellow,every mark/.append style={solid},mark=*\\
    very thick,solid,cbBlue,every mark/.append style={solid},mark=*\\
    very thick,solid,cbVermillion,every mark/.append style={solid},mark=*\\
    very thick,solid,cbReddischPurple,every mark/.append style={solid},mark=*\\
    very thick,solid,black,every mark/.append style={solid},mark=*\\
}
\pgfplotsset{
    legend style={font=\small},
    compat=1.16,
    width=260pt,
    height=140pt,
    legend cell align=left,
    xlabel near ticks,
    ylabel near ticks,
    every axis/.append style={
        cycle list name=cbSafeList,
        ymin=0,
        enlargelimits=0.05,
        mark size=1pt,
        ylabel style={align=center},
        xlabel style={align=center},
        title style={align=center}
    }
}

%% PgfplotsTable: loading data files to use with pgfplots.
\usepackage{pgfplotstable}


%% Support for hyperlinks and urls. The setting ``colorlinks'' sets how links
%% are shown in the document (with a color, without underline). We put hyperref
%% last---it has a tendency to break other packages when loaded before them.
\usepackage[colorlinks]{hyperref}
\usepackage{graphicx}
\usepackage{pgffor}
\usepackage{caption}
\usepackage{tabularx}
\usepackage{enumitem}
\usepackage{fancyhdr}


\begin{document}

%% First the meta-data of the doucment: title,

\begin{titlepage}
    \centering
    \vspace*{3cm}
    
    {\Huge COMPSCI 2DB3 Assignment 2}\\[2cm]
    
    {\large Luca Mawyin}\\[0.5cm]
    
    {\large Dr. Jelle Hellings}\\[0.5cm]

    {\large \today}\\[2cm]

    {\Large McMaster University}

    \vfill
    \begin{flushleft}

        {\large \textbf{Student Number: }400531739}\\[0.5cm]
        {\large \textbf{MacID:}~mawyinl}
    \end{flushleft}

\end{titlepage}

\stepcounter{section}
\section*{Problem \thesection.}

\subsection*{Constraints}
\subsubsection*{Person}
\begin{itemize}
    \item pid is unique 
    \item pid is primary key
\end{itemize}

\subsubsection*{Publisher}
\begin{itemize}
    \item name is primary key
\end{itemize}

\subsubsection*{Book}
\begin{itemize}
    \item ISBN is primary key
    \item Publisher name is foreign key referencing Publisher(name)
\end{itemize}

\subsubsection*{Website}
\begin{itemize}
    \item Weak entity
    \item ISBN is a foreign key referencing Book(ISBN)
    \item url \& ISBN together form primary key
\end{itemize}

\subsubsection*{EBook}
\begin{itemize}
    \item ISA Book
    \item ISBN is primary key
    \item ISBN is foreign key referencing Book(ISBN)
\end{itemize}

\subsubsection*{PrintBook}
\begin{itemize}
    \item ISA Book
    \item ISBN is primary key
    \item ISBN is foreign key referencing Book(ISBN)
    \item EVersion is foreign key referencing EBook(ISBN)
\end{itemize}

\subsubsection*{authorOf}
\begin{itemize}
    \item Relation between Person and Book with attribute rank
    \item pid is foreign key referencing Person(pid)
    \item ISBN is foreign key referencing Book(ISBN)
    \item pid and ISBN together form primary key
\end{itemize}

\subsubsection*{Recipe}
\begin{itemize}
    \item Weak entity
    \item ISBN is foreign key referencing Book(ISBN)
    \item title and ISBN together form primary key
\end{itemize}

\subsubsection*{usedBy}
\begin{itemize}
    \item Recursive relation between Recipe and itself
    \item recipe\_ISBN, recipe\_title is foreign key referencing Recipe(ISBN, title)
    \item ingredient\_ISBN, ingredient\_title is foreign key referencing Recipe(ISBN, title)
    \item recipe\_ISBN, recipe\_title, ingredient\_ISBN, ingredient\_title together form primary key
\end{itemize}

\begin{verbatim}

Person(
    pid PRIMARY KEY,
    name
)

Publisher(
    name PRIMARY KEY
)

Book(
    ISBN PRIMARY KEY,
    title,
    year,
    edition,
    publisher_name,
    FOREIGN KEY (publisher_name) REFERENCES Publisher(name)
)

Website(
    ISBN,
    url,
    PRIMARY KEY (ISBN, url),
    FOREIGN KEY (ISBN) REFERENCES Book(ISBN)
)

EBook(
    ISBN PRIMARY KEY,
    format,
    FOREIGN KEY (ISBN) REFERENCES Book(ISBN)
)

PrintBook(
    ISBN PRIMARY KEY,
    weight,
    type,
    EVersion,
    FOREIGN KEY (ISBN) REFERENCES Book(ISBN),
    FOREIGN KEY (EVersion) REFERENCES EBook(ISBN)
)

authorOf(
    pid,
    ISBN,
    rank,
    PRIMARY KEY (pid, ISBN),
    FOREIGN KEY (pid) REFERENCES Person(pid),
    FOREIGN KEY (ISBN) REFERENCES Book(ISBN)
)

Recipe(
    ISBN,
    title,
    time,
    PRIMARY KEY (ISBN, title),
    FOREIGN KEY (ISBN) REFERENCES Book(ISBN)
)

usedBy(
    recipe_ISBN,
    recipe_title,
    ingredient_ISBN,
    ingredient_title,
    PRIMARY KEY (recipe_ISBN, recipe_title, ingredient_ISBN, ingredient_title),
    FOREIGN KEY (recipe_ISBN, recipe_title) REFERENCES Recipe(ISBN, title),
    FOREIGN KEY (ingredient_ISBN, ingredient_title) REFERENCES Recipe(ISBN, title)
)

\end{verbatim}

\stepcounter{section}
\section*{Problem \thesection.}

\begin{verbatim}
CREATE TABLE Person (
    pid INT PRIMARY KEY,
    name VARCHAR(255) NOT NULL
);

CREATE TABLE Publisher (
    name VARCHAR(255) PRIMARY KEY
);

CREATE TABLE Book (
    ISBN VARCHAR(20) PRIMARY KEY,
    title VARCHAR(255) NOT NULL,
    year INT,
    edition SMALLINT,
    publisher_name VARCHAR(255),
    FOREIGN KEY (publisher_name) REFERENCES Publisher(name)
);

-- Cannot represent weak entity in SQL 
-- Representing Website and Recipe as separate tables with foreign keys referencing Book(ISBN)
CREATE TABLE Website (
    ISBN VARCHAR(20) NOT NULL,
    url VARCHAR(255) NOT NULL,
    PRIMARY KEY (ISBN, url),
    FOREIGN KEY (ISBN) REFERENCES Book(ISBN)
);


-- Cannot represent ISA relationship in SQL 
-- Representing EBook and PrintBook as separate tables with foreign keys referencing Book(ISBN)
CREATE TABLE EBook (
    ISBN VARCHAR(20) PRIMARY KEY,
    format VARCHAR(50),
    FOREIGN KEY (ISBN) REFERENCES Book(ISBN)
);

CREATE TABLE PrintBook (
    ISBN VARCHAR(20) PRIMARY KEY,
    weight INT,
    type VARCHAR(255),
    EVersion VARCHAR(20),
    FOREIGN KEY (ISBN) REFERENCES Book(ISBN),
    FOREIGN KEY (EVersion) REFERENCES EBook(ISBN)
);

CREATE TABLE authorOf (
    pid INT NOT NULL,
    ISBN VARCHAR(20) NOT NULL,
    rank INT,
    PRIMARY KEY (pid, ISBN),
    FOREIGN KEY (pid) REFERENCES Person(pid),
    FOREIGN KEY (ISBN) REFERENCES Book(ISBN)
);

CREATE TABLE Recipe (
    ISBN VARCHAR(20) NOT NULL,
    title VARCHAR(255) NOT NULL,
    time INT,
    PRIMARY KEY (ISBN, title),
    FOREIGN KEY (ISBN) REFERENCES Book(ISBN)
);

CREATE TABLE usedBy (
    recipe_ISBN VARCHAR(20) NOT NULL,
    recipe_title VARCHAR(255) NOT NULL,
    ingredient_ISBN VARCHAR(20) NOT NULL,
    ingredient_title VARCHAR(255) NOT NULL,
    PRIMARY KEY (recipe_ISBN, recipe_title, ingredient_ISBN, ingredient_title),
    FOREIGN KEY (recipe_ISBN, recipe_title) REFERENCES Recipe(ISBN, title),
    FOREIGN KEY (ingredient_ISBN, ingredient_title) REFERENCES Recipe(ISBN, title)
);

\end{verbatim}

\stepcounter{section}
\section*{Problem \thesection.}
The best way to make recipes depend only on recipes of the same book is to only include a single ISBN in the usedBy relation instead of two. As both the recipe and ingredient are from the same book, the ISBN is implied for both attributes. Then, to validate that the ingredient recipe is from the same book as the recipe, it can be queried from the ingredient\_title and the shared ISBN.

\stepcounter{section}
\section*{Problem \thesection.}
In the resulting relational schema, it could be important to create some ISBN standard for the database. This would ensure that all ISBNs within the database are the same length and structure, making any sort of validation or querying easier and less error prone.

\stepcounter{section}
\section*{Problem \thesection.}

As the relational model and SQL statements stand, despite Book being subjected to two subclasses, it is technically able to exist independently, which is not ideal. We want to restrict books to being exclusively either an EBook or a PrintBook. To enforce this constraint in the SQL, we can add a column to the Book table called type, which can only have values of either EBook or PrintBook. Then, a CHECK constraint can be added to the Book table to ensure that the entered values in the type column are valid.

In addition to the Book issues, there is also an issue with the usedBy table. As the usedBy table is a recursive relation on the Recipe table, it is possible for a recipe to reference an ingredient that references the recipe, which creates a cycle. In order to prevent looping, a CHECK constraint can be added to the usedBy table to ensure that the recipe\_title is not the same as the ingredient\_title. Furthermore, there were no established cardinality constraints for the usedBy relation. This means that a recipe could use or be used by other recipes an unlimited number of times.

\stepcounter{section}
\section*{Problem \thesection.}

\begin{verbatim}
CREATE TABLE Review(
    pid INT NOT NULL,
    ISBN VARCHAR(20) NOT NULL,
    score SMALLINT CHECK (score >= 0 and score <= 10), 
    last_updated TIMESTAMP DEFAULT CURRENT_TIMESTAMP,
    description TEXT,
    PRIMARY KEY (pid, ISBN),
    FOREIGN KEY (pid) REFERENCES Person(pid),
    FOREIGN KEY (ISBN) REFERENCES Book(ISBN)
);

CREATE TABLE ReviewVote(
    pid INT NOT NULL,
    rpid INT NOT NULL,
    risbn VARCHAR(20) NOT NULL,
    up BOOLEAN DEFAULT FALSE,
    FOREIGN KEY (rpid, risbn) REFERENCES Review(pid, ISBN),
    FOREIGN KEY (pid) REFERENCES Person(pid),
    PRIMARY KEY (rpid, risbn)
);

CREATE TABLE Recommend(
    pid INT NOT NULL,
    friend_pid INT NOT NULL,
    recipe_ISBN VARCHAR(20) NOT NULL,
    description TEXT,
    FOREIGN KEY (pid) REFERENCES Person(pid),
    FOREIGN KEY (friend_pid) REFERENCES Person(pid),
    FOREIGN KEY (recipe_ISBN) REFERENCES Recipe(ISBN),
    PRIMARY KEY (pid, friend_pid, recipe_ISBN)
);

\end{verbatim}

\stepcounter{section}
\section*{Problem \thesection.}
Given the SQL constructions that we have seen in the course so far, it is not possible to implement the constraint of a author not being able to review their own book. Subsequently, it is also not possible to implement the constraint of a user not being able to recommend a recipe to themselves. Given the SQL constructions we have seen in the course, it is possible to query the database for user information and book information to check if a used is the author of the book that they are reviewing, or if a user is recommending a recipe to themselves. However, it is not possible to enforce these constraints as the data is being entered into the database. This is because there is no conditional logic that would allow us to add a CHECK constraint to the Review and Recommend tables to check if the pid of the review or recommendation is the same as the pid of the author of the book being reviewed or recommended.

\end{document}